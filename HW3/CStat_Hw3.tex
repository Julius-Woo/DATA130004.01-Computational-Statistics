\documentclass[UTF8]{ctexart}
\ctexset { section = { format={\Large \bfseries } } }
\pagestyle{plain}
\usepackage{float}
\usepackage{amsmath}
\usepackage{amssymb}
\usepackage{listings}
\usepackage{graphicx}
\usepackage{xcolor}
\usepackage{geometry}
\geometry{a4paper,scale=0.8}
\usepackage{caption}
\captionsetup[figure]{name={Figure}}
\captionsetup[table]{name={Table}}

\lstset{tabsize=4,
language=R, %
frame=none,  %把代码用带有阴影的框圈起来
backgroundcolor=\color[RGB]{240,240,240},%背景
commentstyle=\color{black!50!white},
rulesepcolor=\color{red!20!green!20!blue!20},  %代码块边框为淡青色
keywordstyle=\color{blue},  %代码关键字的颜色为蓝色,粗体
showstringspaces=false,  %不显示代码字符串中间的空格标记
stringstyle=\ttfamily,  % 代码字符串的特殊格式
keepspaces=true,
breakindent=22pt,
stepnumber=1,
basicstyle={\footnotesize\ttfamily},
showspaces=false,
flexiblecolumns=true,
breaklines=true,  %对过长的代码自动换行
breakautoindent=true,
breakindent=4em,
aboveskip=1em,  %代码块边框
fontadjust,
captionpos=t,
framextopmargin=1pt,framexbottommargin=1pt,abovecaptionskip=-9pt,belowcaptionskip=9pt,
xleftmargin=2em,xrightmargin=1em,  % 设定listing左右的空白
texcl=true,
% 设定中文冲突,断行,列模式,数学环境输入,listing数字的样式
extendedchars=false,columns=flexible,mathescape=false
numbersep=-1em,
morekeywords={in, TRUE, FALSE, filter, group, color, size}
}

\title{\textbf{Computational Statistics Homework 3}}
\author{吴嘉骜 21307130203}
\date{\today}

\begin{document}

\maketitle

\noindent
\section{ex 5.11}
\setlength{\parindent}{0pt}
If $\hat{\theta}_1$ and $\hat{\theta}_2$ are unbiased estimators of $\theta$, 
and $\hat{\theta}_1$ and $\hat{\theta}_2$ are antithetic, we derived that $c^* = 1/2$ 
is the optimal constant that minimizes the variance of $\hat{\theta}_c = c\hat{\theta}_1 + (1-c)\hat{\theta}_2$.
Derive $c^*$ for the general case. That is, if $\hat{\theta}_1$ and $\hat{\theta}_2$ are unbiased estimators of $\theta$, 
find the value $c^*$ that minimizes the variance of $\hat{\theta}_c = c\hat{\theta}_1 + (1-c)\hat{\theta}_2$
in equation (5.11). ($c^*$ will be a function of the variances and the covariance of the estimators.)\\
\textbf{Solution}:\\
If $\hat{\theta}_1$ and $\hat{\theta}_2$ are any unbiased estimators of $\theta$, we have
\begin{align*}
    Var(\hat{\theta}_c) &= c^2Var(\hat{\theta}_1) + (1-c)^2Var(\hat{\theta}_2) + 2c(1-c)Cov(\hat{\theta}_1, \hat{\theta}_2)\\
    &= (Var(\hat{\theta}_1) + Var(\hat{\theta}_2) - 2Cov(\hat{\theta}_1, \hat{\theta}_2))c^2 + 2(Cov(\hat{\theta}_1, \hat{\theta}_2) - Var(\hat{\theta}_2))c + Var(\hat{\theta}_2)
\end{align*}
This is a quadratic function of $c$, and the minimum is achieved at
\begin{align*}
    c^* = \frac{Var(\hat{\theta}_2) - Cov(\hat{\theta}_1, \hat{\theta}_2)}{Var(\hat{\theta}_1) + Var(\hat{\theta}_2) - 2Cov(\hat{\theta}_1, \hat{\theta}_2)}.
\end{align*}


\section{}
Consider estimating the rare event probability $P(Z > 10)$ where $Z \sim N(0, 1)$.\\
(a) Apply the simple Monte Carlo method, record your results when the sample size
is $10^3, 10^4, \ldots$\\
(b) Apply the change of measure using $Y \sim N(10, 1)$, record your results when the
sample size is $10^3, 10^4, \ldots$\\
\textbf{Solution}:\\
The R code is as follows:
\begin{lstlisting}
    # Rare event P(Z>10) where Z~N(0,1)
    # (a) Simple Monte Carlo
    SMC <- function(N){
      X <- rnorm(N)
      mean(X>10)
    }
    
    # (b) Change of measure
    CoM <- function(N){
      Y <- rnorm(N, mean=10, sd=1)
      mean((Y>10)*exp(50-10*Y))
    }
    
    
    Ns <- c(1e3,1e4,1e5,1e6,1e7)
    results <- matrix(NA, nrow=length(Ns), ncol=3)
    colnames(results) <- c("Sample Size", "Simple Monte Carlo", "Change of Measure")
    
    for (i in 1:length(Ns)){
      n <- Ns[i]
      results[i,] <- c(n, SMC(n), CoM(n))
    }
    results_df <- as.data.frame(results)
    results_df
\end{lstlisting}
The results are as follows:
\begin{lstlisting}
> results_df
  Sample Size Simple Monte Carlo Change of Measure
1       1e+03                  0      9.366175e-24
2       1e+04                  0      7.532025e-24
3       1e+05                  0      7.633165e-24
4       1e+06                  0      7.642613e-24
5       1e+07                  0      7.625574e-24
\end{lstlisting}
From the results, we can see that the simple Monte Carlo method cannot give a good estimation of the rare event probability,
because the probability $P(Z > 10), \ Z \sim N(0, 1)$ is too small, and a much larger sample size is needed to get a good estimation. By the way, my hardware cannot afford a sample size larger than $10^9$.\\
However, the change of measure method can present a satisfactory estimation, because $P(Y>10), \ Y \sim N(10,1)$ is much larger,
and exponential function is continuous, giving small but non-zero values for us to calculate.

\end{document}