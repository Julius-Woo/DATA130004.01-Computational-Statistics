\documentclass[UTF8]{ctexart}
\ctexset { section = { format={\Large \bfseries } } }
\pagestyle{plain}
\usepackage{float}
\usepackage{amsmath}
\usepackage{amssymb}
\usepackage{listings}
\usepackage{graphicx}
\usepackage{xcolor}
\usepackage{geometry}
\geometry{a4paper,scale=0.8}
\usepackage{caption}
\captionsetup[figure]{name={Figure}}
\captionsetup[table]{name={Table}}

\lstset{tabsize=4,
language=R, %
frame=none,  %把代码用带有阴影的框圈起来
backgroundcolor=\color[RGB]{240,240,240},%背景
commentstyle=\color{black!50!white},
rulesepcolor=\color{red!20!green!20!blue!20},  %代码块边框为淡青色
keywordstyle=\color{blue},  %代码关键字的颜色为蓝色,粗体
showstringspaces=false,  %不显示代码字符串中间的空格标记
stringstyle=\ttfamily,  % 代码字符串的特殊格式
keepspaces=true,
breakindent=22pt,
stepnumber=1,
basicstyle={\footnotesize\ttfamily},
showspaces=false,
flexiblecolumns=true,
breaklines=true,  %对过长的代码自动换行
breakautoindent=true,
breakindent=4em,
aboveskip=1em,  %代码块边框
fontadjust,
captionpos=t,
framextopmargin=1pt,framexbottommargin=1pt,abovecaptionskip=-9pt,belowcaptionskip=9pt,
xleftmargin=2em,xrightmargin=1em,  % 设定listing左右的空白
texcl=true,
% 设定中文冲突,断行,列模式,数学环境输入,listing数字的样式
extendedchars=false,columns=flexible,mathescape=false
numbersep=-1em,
morekeywords={in, TRUE, FALSE, filter, group, color, size}
}

\title{\textbf{Computational Statistics Homework 2}}
\author{吴嘉骜 21307130203}
\date{\today}

\begin{document}

\maketitle

\noindent
\section{ex 5.2}
\setlength{\parindent}{0pt}
Refer to Example 5.3.
Compute a Monte Carlo estimate of the standard
normal cdf, by generating from the Uniform$(0,x)$ distribution. Compare your
estimates with the normal cdf function \texttt{pnorm}. Compute an estimate of the
variance of your Monte Carlo estimate of $\Phi(2)$, and a 95\% confidence interval
for $\Phi(2)$.\\
\textbf{Solution:}\\
The R code and result are as follows:
\begin{lstlisting}
    # Monte Carlo estimate of the standard normal cdf
    x <- seq(0.1,2.5,length=10)
    m <- 10000
    cdf <- numeric(length(x))
    for(i in 1:length(x)){
        j <- runif(m,0,x[i])
        A <- exp(-j^2/2)
        cdf[i] <- 0.5 + x[i]*mean(A)/sqrt(2*pi)
    }
    Phi <- pnorm(x)
    print(round(rbind(x,cdf,Phi),3))


    > print(round(rbind(x,cdf,Phi),3))
        [,1]  [,2]  [,3]  [,4]  [,5]  [,6]  [,7]  [,8]  [,9] [,10]
    x   0.10 0.367 0.633 0.900 1.167 1.433 1.700 1.967 2.233 2.500
    cdf 0.54 0.643 0.737 0.816 0.878 0.923 0.955 0.977 0.983 0.993
    Phi 0.54 0.643 0.737 0.816 0.878 0.924 0.955 0.975 0.987 0.994
\end{lstlisting}
From the result, we can see that the Monte Carlo estimate is very close to the normal cdf function \texttt{pnorm}.\\
To estimate the variance of the Monte Carlo estimate of $\theta = \Phi(2)$, notice that the estimator $\hat{\theta}$ is computed by 
$\hat{\theta} = 0.5 + \frac{x}{m}\sum\limits_{i=1}^{m}g(x_i)$, where $g(t) = \frac{1}{\sqrt{2\pi}}e^{-t^2/2}$ and $x_i \sim$ Uniform(0,$x$), iid.\\
So $Var(\hat{\theta}) = \frac{x^2}{m} Var(g(x))$. We can estimate $Var(g(x))$ by the sample variance of $g(x_i)$, $i = 1,2,\ldots,m$.\\
To compute a 95\% confidence interval for $\theta$, by the Central Limit Theorem,
we have $\hat{\theta} \sim N(\theta, Var(\hat{\theta}))$ approximately for large $m$. Then an approximate 95\% confidence interval for $\theta$ is given by
\begin{equation*}
    \hat{\theta} \pm z_{0.025}\sqrt{\widehat{Var(\hat{\theta})}}.
\end{equation*}
we can use the following R code:
\begin{lstlisting}
    # an estimate of the variance of Monte Carlo estimate of Φ(2)
    x <- 2
    m <- 10000
    j <- runif(m,0,x)
    g <- exp(-j^2/2)
    theta <- x*mean(g)/sqrt(2*pi) + 0.5
    varg <- mean((g - mean(g))^2)
    varhat <- x^2*varg/(m*2*pi)
    c(theta, varhat)
    pnorm(2)
    # a 95\% confidence interval for Φ(2)
    theta + qnorm(c(0.025, 0.975)) * sqrt(varhat)

    > c(theta, varhat)
    [1] 9.770137e-01 5.327914e-06
    > pnorm(2)
    [1] 0.9772499
    > # a 95\% confidence interval for Φ(2)
    > theta + qnorm(c(0.025, 0.975)) * sqrt(varhat)
    [1] 0.9724897 0.9815377
\end{lstlisting}


\section{ex 5.3}
Compute a Monte Carlo estimate $\hat{\theta}$ of 
\begin{equation*}
    \theta = \int_{0}^{0.5} e^{-x} dx
\end{equation*}
by sampling from Uniform(0, 0.5), and estimate the variance of $\hat{\theta}$. Find another 
Monte Carlo estimator $\theta^∗$ by sampling from the exponential distribution.
Which of the variances (of $\hat{\theta}$ and $\theta^∗$) is smaller, and why?\\
\textbf{Solution:}\\
First we estimate $\theta$ by sampling from Uniform(0, 0.5). The R code and results are as follows:
\begin{lstlisting}
    # Monte Carlo estimate from Uniform(0, 0.5).
    m <- 10000
    u <- runif(m,0,0.5)
    g <- exp(-u)
    theta <- 0.5*mean(g)
    theta
    1 - exp(-0.5)
    # estimate variance1
    var1 <- 0.5*0.5* mean((g-mean(g))^2)/m
    var1
    
    > theta
    [1] 0.3935569
    > 1 - exp(-0.5)
    [1] 0.3934693
    > # estimate variance1
    > var1 <- 0.5*0.5* mean((g-mean(g))^2)/m
    > var1
    [1] 3.221585e-07
\end{lstlisting}
From the result, we can see that the Monte Carlo estimate $\hat{\theta}$ is very close to the exact value.\\
Then we estimate $\theta$ by sampling from the exponential distribution. The R code and results are as follows:
\begin{lstlisting}
    # sampling from the exponential distribution
    m <- 10000
    v <- rexp(m,1)
    theta <- mean(v<=0.5)
    theta
    p <- 1 - exp(-0.5)
    p
    # estimate variance2
    var2 <- theta*(1-theta)/m
    var2
    accvar <- p*(1-p)/m
    accvar

    > theta
    [1] 0.3938
    > p <- 1 - exp(-0.5)
    > p
    [1] 0.3934693
    > # estimate variance2
    > var2 <- theta*(1-theta)/m
    > var2
    [1] 2.387216e-05
    > accvar <- p*(1-p)/m
    > accvar
    [1] 2.386512e-05
\end{lstlisting}
We compare the variances of $\hat{\theta}$ and $\theta^∗$ as follows:
\begin{lstlisting}
    > var1/var2
    [1] 0.01349516
\end{lstlisting}
The result shows that $Var(\hat{\theta})<Var(\theta^∗)$. 
The reason is that the exact value of $Var(\hat{\theta})$ is smaller than the accurate value of $Var(\theta^∗)$, which is $\theta(1-\theta)/m = 2.386512e-05$.\\
Now we calculate the exact value of $Var(\hat{\theta})$ to clarify the reason.\\
Var($\hat{\theta}$) = $\frac{0.5^2}{m}Var(g(U))$, where $U \sim$ Uniform(0,0.5).\\
\begin{equation*}
    \begin{split}
        Var(g(U)) &= E(g^2(U)) - E(g(U))^2\\
        &= \int_{0}^{0.5} 2e^{-2x} dx - (\int_{0}^{0.5} 2e^{-x} dx)^2\\
        &= 1 - e^{-1} - 4(1 - e^{-0.5})^2\\
        &= 8e^{-0.5} - 5e^{-1} -3\\
    \end{split}
\end{equation*}
So $Var(\hat{\theta}) = \frac{0.5^2}{m}(8e^{-0.5} - 5e^{-1} -3) = 3.212018e − 07$.\\
Then $\frac{Var(\hat{\theta})}{Var(\theta^∗)} = 0.01345905$.\\


\section{ex 5.6}
In Example 5.7 the control variate approach was illustrated for Monte Carlo integration of
\begin{equation*}
    \theta = \int_{0}^{1} e^{x} dx
\end{equation*}
Now consider the antithetic variate approach. Compute $Cov(e^U, e^{1−U})$ and
$Var(e^U + e^{1−U})$, where $U \sim$ Uniform(0,1). What is the percent reduction in
variance of $\hat{\theta}$ that can be achieved using antithetic variates (compared with
simple MC)?\\
\textbf{Solution:}\\
$Cov(e^U, e^{1−U}) = E(e^Ue^{1−U}) - E(e^U)E(e^{1−U}) = e - (e-1)^2 = −0.2342106$.\\
$Var(e^U) = Var(e^{1−U})=E(e^{2U}) - (E(e^U))^2 = 0.5(e^2-1) - (e-1)^2 = 0.2420356$.\\
$Var(e^U + e^{1−U}) = 2Var(e^U)+2Cov(e^U, e^{1−U}) = e^2-1-4(e-1)^2+2e = 0.01564999$.\\
If we use the antithetic variate approach, we have
$Var(\hat{\theta}_{ant}) = Var(\frac{1}{2}(e^U + e^{1−U})) = 0.003912497$.\\
Otherwise, using the simple Monte Carlo method, we have $Var(\hat{\theta}) = Var(\frac{1}{2}(e^U+e^V)) = \frac{1}{2}Var(e^U)= 0.1210178$, where $V \sim$ Uniform(0,1) and $U$ and $V$ are independent.\\
Then the percent reduction in variance that can be achieved using antithetic variates is
\begin{equation*}
    \frac{Var(\hat{\theta}) - Var(\hat{\theta}_{ant})}{Var(\hat{\theta})}  = 0.96767 = 96.767\%.
\end{equation*}


\section{ex 5.7}
Refer to Exercise 5.6. Use a Monte Carlo simulation to estimate $\theta$ by the
antithetic variate approach and by the simple Monte Carlo method. Compute
an empirical estimate of the percent reduction in variance using the antithetic
variate. Compare the result with the theoretical value from Exercise 5.6.\\
\textbf{Solution:}\\
The R code and results are as follows:
\begin{lstlisting}
    # estimate theta by the antithetic variate approach
    m <- 10000
    u <- runif(m/2)
    v <- 1-u
    g1 <- (exp(u)+exp(v))/2
    anti <- mean(g1)
    v1 <- mean((g1-anti)^2)/(m/2)
    
    # estimate theta by the simple Monte Carlo method
    u2 <- runif(m)
    g2 <- exp(u2)
    smc <- mean(g2)
    v2 <- mean((g2-smc)^2)/m
    
    c(anti, smc)
    c(v1,v2)
    # reduction
    (v2 - v1)/v2

    > c(anti, smc)
    [1] 1.717577 1.709982
    > c(v1,v2)
    [1] 7.730742e-07 2.387231e-05
    > # reduction
    > (v2 - v1)/v2
    [1] 0.9676163
\end{lstlisting}
The result shows that the empirical estimate of the percent reduction in variance is 96.762\%, which is very close to the theoretical value 96.767\% in Exercise 5.6.\\

\section{ex 5.8}
Let $U \sim$ Uniform(0,1), $X = aU$, and $X^{\prime} = a(1 − U)$, where $a$ is a constant.
Show that $\rho(X, X^{\prime}) = −1$. Is $\rho(X, X^{\prime}) = −1$ if $U$ is a symmetric beta random
variable?\\
\textbf{Proof:}\\
For all $U$ with finite mean and variance, let $\mu = E(U)$, $\sigma^2 = Var(U)$.\\
$Var(X) = Var(aU) = a^2Var(U) = a^2\sigma^2$.\\
$Var(X^{\prime}) = Var(a(1-U)) = a^2Var(1-U) = a^2Var(U) = a^2\sigma^2$.\\
$Cov(X, X^{\prime}) = Cov(aU, a(1-U)) = a^2Cov(U, 1-U) = a^2[E(U(1-U))-E(U)E(1-U)] = a^2[\mu-(\mu^2+\sigma^2)-\mu(1-\mu)]=-a^2\sigma^2$.\\
$\rho(X, X^{\prime}) = \frac{Cov(X, X^{\prime})}{\sqrt{Var(X)Var(X^{\prime})}} = -1$.\\
In particular, the conclusion holds when $U \sim$ Uniform(0,1) or $U$ is a symmetric beta random variable.\\

\section{ex 5.10}
Use Monte Carlo integration with antithetic variables to estimate
\begin{equation*}
    \theta = \int_{0}^{1} \frac{e^{-x}}{1+x^2} dx,
\end{equation*}
and find the approximate reduction in variance as a percentage of the variance
without variance reduction.\\
\textbf{Solution:}\\
The R code and results are as follows:
\begin{lstlisting}
    # estimate θ by the antithetic variate approach
    m <- 10000
    u <- runif(m/2)
    v <- 1-u
    g1 <- (exp(-u)/(1 + u^2)+exp(-v)/(1 + v^2))/2
    anti <- mean(g1)
    v1 <- mean((g1-anti)^2)/(m/2)
    
    # estimate θ by the simple Monte Carlo method
    u2 <- runif(m)
    g2 <- exp(-u2)/(1 + u2^2)
    smc <- mean(g2)
    v2 <- mean((g2-smc)^2)/m
    
    c(anti, smc)
    c(v1,v2)
    # reduction
    (v2 - v1)/v2

    > c(anti, smc)
    [1] 0.5243512 0.5232506
    > c(v1,v2)
    [1] 2.142252e-07 6.061683e-06
    > # reduction
    > (v2 - v1)/v2
    [1] 0.9646591
\end{lstlisting}
The approximate reduction in variance as a percentage of the variance without variance
reduction is 96.4659\%.\\

\section{}
Monte Carlo method can be used to approximate the fraction of a $d$-dimensional
hypersphere which lies in the inscribed $d$-dimensional hypercube.
Simulate with different dimensions $d$ = 2, 3, 4, $\ldots$ , 10. (Hint: use \texttt{apply} function.)\\
(1) Derive the formula for the EXACT values for the above problem for each $d$-dimension.\\
(2) Using the above formula, approximate the value of $\pi$.
Find the sample size needed to approximate $\pi$ to its 5th digit, i.e., the first time when you have
an estimate as 3.14159x, for each dimension $d$.
Set the random seed with \texttt{set.seed(123)} at the beginning of your R code.\\
\textbf{Solution:}\\
(1) The volume of a $d$-dimensional hypersphere with radius $r$ is given by:
\begin{equation*}
    V_d = \frac{\pi^{d/2}}{\Gamma(d/2+1)}r^d.
\end{equation*}
where $\Gamma$ is the gamma function.\\
We assume that the hypersphere is inscribed in a hypercube with side length 1, then $r=0.5$.\\
The volume of a $d$-dimensional hypercube with side length $l$ is $V_d^{\prime} = l^d$, and for our problem, $l=1$.\\
The fraction $f_d$​ of the d-dimensional hypersphere that lies in the inscribed d-dimensional hypercube is:\\
$f_d = \frac{V_d}{V_d^{\prime}} = \frac{\pi^{d/2}}{\Gamma(d/2+1)}2^{-d}$,\\
which is the desired formula.\\
(2) Using the method of Monte Carlo, we can approximate $\pi$ by randomly generating points in the d-dimensional hypercube and 
checking how many of those points lie within the hypersphere. The ratio of points inside the hypersphere to the total number of points gives an approximation of 
$f_d$, which we can then use to solve for $\pi$.\\
The R code is as follows:
\begin{lstlisting}
    set.seed(123)

    # use Monte Carlo to approximate pi for dimension d
    # N: total sample number
    # prevcount: previous count
    mcpi <- function(d,N,prevcount=0){
      count <- prevcount
      
      if (count==0){
        for(i in 1:N){
          u <- runif(d,-0.5,0.5)
          if(sum(u^2)<=0.5*0.5)
            count <- count+1
        }
      }
      # skip the previous samples
      else{
        u <- runif(d,-0.5,0.5)
        if(sum(u^2)<=0.5*0.5)
          count <- count+1
      }
      f <- count/N
      pi_est <- (f * 2^d * gamma(d/2 + 1))^(2/d)
      return(list(pi_est = pi_est, count = count))
    }
    
    # determine the minimum n to approximate π to its 5th digit
    sizeto5 <- function(d){
      N <- 2000
      prevresult <- mcpi(d,N)
      while (TRUE) {
        if (trunc(prevresult$pi_est*1e5) == 314159)
          return(c(Dimension = d, Sample_Size = N, Pi_Approximation = round(prevresult$pi_est, 6)))
        N <- N+1
        prevresult <- mcpi(d, N, prevresult$count)
      }
    }
    
    dims <- 2:10
    results <- lapply(dims, sizeto5)
    
    results_df <- as.data.frame(do.call(rbind, results))
    print(results_df)
\end{lstlisting}
In the code above, the function \texttt{mcpi} is used to approximate $\pi$ for dimension $d$, and the function \texttt{sizeto5} is used to determine the minimum $n$ to approximate $\pi$ to its 5th digit. Empirically, we
choose to start with $N = 2000$ samples, and then increase $N$ by 1 until we get the desired approximation. The results are as follows:
\begin{lstlisting}
    > print(results_df)
    Dimension Sample_Size Pi_Approximation
  1         2        9753         3.141597
  2         3        4089         3.141597
  3         4       20141         3.141595
  4         5       37126         3.141596
  5         6        4025         3.141590
  6         7       23461         3.141595
  7         8        5109         3.141594
  8         9        5588         3.141590
  9        10        5220         3.141599
\end{lstlisting}

\end{document}